\documentclass[12pt]{article}
\usepackage{fancyhdr}
\usepackage{color}
\definecolor{AliceBlue}{rgb}{0.94,0.97,1.00}
\definecolor{AntiqueWhite1}{rgb}{1.00,0.94,0.86}
\definecolor{AntiqueWhite2}{rgb}{0.93,0.87,0.80}
\definecolor{AntiqueWhite3}{rgb}{0.80,0.75,0.69}
\definecolor{AntiqueWhite4}{rgb}{0.55,0.51,0.47}
\definecolor{AntiqueWhite}{rgb}{0.98,0.92,0.84}
\definecolor{BlanchedAlmond}{rgb}{1.00,0.92,0.80}
\definecolor{BlueViolet}{rgb}{0.54,0.17,0.89}
\definecolor{CadetBlue1}{rgb}{0.60,0.96,1.00}
\definecolor{CadetBlue2}{rgb}{0.56,0.90,0.93}
\definecolor{CadetBlue3}{rgb}{0.48,0.77,0.80}
\definecolor{CadetBlue4}{rgb}{0.33,0.53,0.55}
\definecolor{CadetBlue}{rgb}{0.37,0.62,0.63}
\definecolor{CornflowerBlue}{rgb}{0.39,0.58,0.93}
\definecolor{DarkBlue}{rgb}{0.00,0.00,0.55}
\definecolor{DarkCyan}{rgb}{0.00,0.55,0.55}
\definecolor{DarkGoldenrod1}{rgb}{1.00,0.73,0.06}
\definecolor{DarkGoldenrod2}{rgb}{0.93,0.68,0.05}
\definecolor{DarkGoldenrod3}{rgb}{0.80,0.58,0.05}
\definecolor{DarkGoldenrod4}{rgb}{0.55,0.40,0.03}
\definecolor{DarkGoldenrod}{rgb}{0.72,0.53,0.04}
\definecolor{DarkGray}{rgb}{0.66,0.66,0.66}
\definecolor{DarkGreen}{rgb}{0.00,0.39,0.00}
\definecolor{DarkGrey}{rgb}{0.66,0.66,0.66}
\definecolor{DarkKhaki}{rgb}{0.74,0.72,0.42}
\definecolor{DarkMagenta}{rgb}{0.55,0.00,0.55}
\definecolor{DarkOliveGreen1}{rgb}{0.79,1.00,0.44}
\definecolor{DarkOliveGreen2}{rgb}{0.74,0.93,0.41}
\definecolor{DarkOliveGreen3}{rgb}{0.64,0.80,0.35}
\definecolor{DarkOliveGreen4}{rgb}{0.43,0.55,0.24}
\definecolor{DarkOliveGreen}{rgb}{0.33,0.42,0.18}
\definecolor{DarkOrange1}{rgb}{1.00,0.50,0.00}
\definecolor{DarkOrange2}{rgb}{0.93,0.46,0.00}
\definecolor{DarkOrange3}{rgb}{0.80,0.40,0.00}
\definecolor{DarkOrange4}{rgb}{0.55,0.27,0.00}
\definecolor{DarkOrange}{rgb}{1.00,0.55,0.00}
\definecolor{DarkOrchid1}{rgb}{0.75,0.24,1.00}
\definecolor{DarkOrchid2}{rgb}{0.70,0.23,0.93}
\definecolor{DarkOrchid3}{rgb}{0.60,0.20,0.80}
\definecolor{DarkOrchid4}{rgb}{0.41,0.13,0.55}
\definecolor{DarkOrchid}{rgb}{0.60,0.20,0.80}
\definecolor{DarkRed}{rgb}{0.55,0.00,0.00}
\definecolor{DarkSalmon}{rgb}{0.91,0.59,0.48}
\definecolor{DarkSeaGreen1}{rgb}{0.76,1.00,0.76}
\definecolor{DarkSeaGreen2}{rgb}{0.71,0.93,0.71}
\definecolor{DarkSeaGreen3}{rgb}{0.61,0.80,0.61}
\definecolor{DarkSeaGreen4}{rgb}{0.41,0.55,0.41}
\definecolor{DarkSeaGreen}{rgb}{0.56,0.74,0.56}
\definecolor{DarkSlateBlue}{rgb}{0.28,0.24,0.55}
\definecolor{DarkSlateGray1}{rgb}{0.59,1.00,1.00}
\definecolor{DarkSlateGray2}{rgb}{0.55,0.93,0.93}
\definecolor{DarkSlateGray3}{rgb}{0.47,0.80,0.80}
\definecolor{DarkSlateGray4}{rgb}{0.32,0.55,0.55}
\definecolor{DarkSlateGray}{rgb}{0.18,0.31,0.31}
\definecolor{DarkSlateGrey}{rgb}{0.18,0.31,0.31}
\definecolor{DarkTurquoise}{rgb}{0.00,0.81,0.82}
\definecolor{DarkViolet}{rgb}{0.58,0.00,0.83}
\definecolor{DeepPink1}{rgb}{1.00,0.08,0.58}
\definecolor{DeepPink2}{rgb}{0.93,0.07,0.54}
\definecolor{DeepPink3}{rgb}{0.80,0.06,0.46}
\definecolor{DeepPink4}{rgb}{0.55,0.04,0.31}
\definecolor{DeepPink}{rgb}{1.00,0.08,0.58}
\definecolor{DeepSkyBlue1}{rgb}{0.00,0.75,1.00}
\definecolor{DeepSkyBlue2}{rgb}{0.00,0.70,0.93}
\definecolor{DeepSkyBlue3}{rgb}{0.00,0.60,0.80}
\definecolor{DeepSkyBlue4}{rgb}{0.00,0.41,0.55}
\definecolor{DeepSkyBlue}{rgb}{0.00,0.75,1.00}
\definecolor{DimGray}{rgb}{0.41,0.41,0.41}
\definecolor{DimGrey}{rgb}{0.41,0.41,0.41}
\definecolor{DodgerBlue1}{rgb}{0.12,0.56,1.00}
\definecolor{DodgerBlue2}{rgb}{0.11,0.53,0.93}
\definecolor{DodgerBlue3}{rgb}{0.09,0.45,0.80}
\definecolor{DodgerBlue4}{rgb}{0.06,0.31,0.55}
\definecolor{DodgerBlue}{rgb}{0.12,0.56,1.00}
\definecolor{FloralWhite}{rgb}{1.00,0.98,0.94}
\definecolor{ForestGreen}{rgb}{0.13,0.55,0.13}
\definecolor{GhostWhite}{rgb}{0.97,0.97,1.00}
\definecolor{GreenYellow}{rgb}{0.68,1.00,0.18}
\definecolor{HotPink1}{rgb}{1.00,0.43,0.71}
\definecolor{HotPink2}{rgb}{0.93,0.42,0.65}
\definecolor{HotPink3}{rgb}{0.80,0.38,0.56}
\definecolor{HotPink4}{rgb}{0.55,0.23,0.38}
\definecolor{HotPink}{rgb}{1.00,0.41,0.71}
\definecolor{IndianRed1}{rgb}{1.00,0.42,0.42}
\definecolor{IndianRed2}{rgb}{0.93,0.39,0.39}
\definecolor{IndianRed3}{rgb}{0.80,0.33,0.33}
\definecolor{IndianRed4}{rgb}{0.55,0.23,0.23}
\definecolor{IndianRed}{rgb}{0.80,0.36,0.36}
\definecolor{LavenderBlush1}{rgb}{1.00,0.94,0.96}
\definecolor{LavenderBlush2}{rgb}{0.93,0.88,0.90}
\definecolor{LavenderBlush3}{rgb}{0.80,0.76,0.77}
\definecolor{LavenderBlush4}{rgb}{0.55,0.51,0.53}
\definecolor{LavenderBlush}{rgb}{1.00,0.94,0.96}
\definecolor{LawnGreen}{rgb}{0.49,0.99,0.00}
\definecolor{LemonChiffon1}{rgb}{1.00,0.98,0.80}
\definecolor{LemonChiffon2}{rgb}{0.93,0.91,0.75}
\definecolor{LemonChiffon3}{rgb}{0.80,0.79,0.65}
\definecolor{LemonChiffon4}{rgb}{0.55,0.54,0.44}
\definecolor{LemonChiffon}{rgb}{1.00,0.98,0.80}
\definecolor{LightBlue1}{rgb}{0.75,0.94,1.00}
\definecolor{LightBlue2}{rgb}{0.70,0.87,0.93}
\definecolor{LightBlue3}{rgb}{0.60,0.75,0.80}
\definecolor{LightBlue4}{rgb}{0.41,0.51,0.55}
\definecolor{LightBlue}{rgb}{0.68,0.85,0.90}
\definecolor{LightCoral}{rgb}{0.94,0.50,0.50}
\definecolor{LightCyan1}{rgb}{0.88,1.00,1.00}
\definecolor{LightCyan2}{rgb}{0.82,0.93,0.93}
\definecolor{LightCyan3}{rgb}{0.71,0.80,0.80}
\definecolor{LightCyan4}{rgb}{0.48,0.55,0.55}
\definecolor{LightCyan}{rgb}{0.88,1.00,1.00}
\definecolor{LightGoldenrod1}{rgb}{1.00,0.93,0.55}
\definecolor{LightGoldenrod2}{rgb}{0.93,0.86,0.51}
\definecolor{LightGoldenrod3}{rgb}{0.80,0.75,0.44}
\definecolor{LightGoldenrod4}{rgb}{0.55,0.51,0.30}
\definecolor{LightGoldenrodYellow}{rgb}{0.98,0.98,0.82}
\definecolor{LightGoldenrod}{rgb}{0.93,0.87,0.51}
\definecolor{LightGray}{rgb}{0.83,0.83,0.83}
\definecolor{LightGreen}{rgb}{0.56,0.93,0.56}
\definecolor{LightGrey}{rgb}{0.83,0.83,0.83}
\definecolor{LightPink1}{rgb}{1.00,0.68,0.73}
\definecolor{LightPink2}{rgb}{0.93,0.64,0.68}
\definecolor{LightPink3}{rgb}{0.80,0.55,0.58}
\definecolor{LightPink4}{rgb}{0.55,0.37,0.40}
\definecolor{LightPink}{rgb}{1.00,0.71,0.76}
\definecolor{LightSalmon1}{rgb}{1.00,0.63,0.48}
\definecolor{LightSalmon2}{rgb}{0.93,0.58,0.45}
\definecolor{LightSalmon3}{rgb}{0.80,0.51,0.38}
\definecolor{LightSalmon4}{rgb}{0.55,0.34,0.26}
\definecolor{LightSalmon}{rgb}{1.00,0.63,0.48}
\definecolor{LightSeaGreen}{rgb}{0.13,0.70,0.67}
\definecolor{LightSkyBlue1}{rgb}{0.69,0.89,1.00}
\definecolor{LightSkyBlue2}{rgb}{0.64,0.83,0.93}
\definecolor{LightSkyBlue3}{rgb}{0.55,0.71,0.80}
\definecolor{LightSkyBlue4}{rgb}{0.38,0.48,0.55}
\definecolor{LightSkyBlue}{rgb}{0.53,0.81,0.98}
\definecolor{LightSlateBlue}{rgb}{0.52,0.44,1.00}
\definecolor{LightSlateGray}{rgb}{0.47,0.53,0.60}
\definecolor{LightSlateGrey}{rgb}{0.47,0.53,0.60}
\definecolor{LightSteelBlue1}{rgb}{0.79,0.88,1.00}
\definecolor{LightSteelBlue2}{rgb}{0.74,0.82,0.93}
\definecolor{LightSteelBlue3}{rgb}{0.64,0.71,0.80}
\definecolor{LightSteelBlue4}{rgb}{0.43,0.48,0.55}
\definecolor{LightSteelBlue}{rgb}{0.69,0.77,0.87}
\definecolor{LightYellow1}{rgb}{1.00,1.00,0.88}
\definecolor{LightYellow2}{rgb}{0.93,0.93,0.82}
\definecolor{LightYellow3}{rgb}{0.80,0.80,0.71}
\definecolor{LightYellow4}{rgb}{0.55,0.55,0.48}
\definecolor{LightYellow}{rgb}{1.00,1.00,0.88}
\definecolor{LimeGreen}{rgb}{0.20,0.80,0.20}
\definecolor{MediumAquamarine}{rgb}{0.40,0.80,0.67}
\definecolor{MediumBlue}{rgb}{0.00,0.00,0.80}
\definecolor{MediumOrchid1}{rgb}{0.88,0.40,1.00}
\definecolor{MediumOrchid2}{rgb}{0.82,0.37,0.93}
\definecolor{MediumOrchid3}{rgb}{0.71,0.32,0.80}
\definecolor{MediumOrchid4}{rgb}{0.48,0.22,0.55}
\definecolor{MediumOrchid}{rgb}{0.73,0.33,0.83}
\definecolor{MediumPurple1}{rgb}{0.67,0.51,1.00}
\definecolor{MediumPurple2}{rgb}{0.62,0.47,0.93}
\definecolor{MediumPurple3}{rgb}{0.54,0.41,0.80}
\definecolor{MediumPurple4}{rgb}{0.36,0.28,0.55}
\definecolor{MediumPurple}{rgb}{0.58,0.44,0.86}
\definecolor{MediumSeaGreen}{rgb}{0.24,0.70,0.44}
\definecolor{MediumSlateBlue}{rgb}{0.48,0.41,0.93}
\definecolor{MediumSpringGreen}{rgb}{0.00,0.98,0.60}
\definecolor{MediumTurquoise}{rgb}{0.28,0.82,0.80}
\definecolor{MediumVioletRed}{rgb}{0.78,0.08,0.52}
\definecolor{MidnightBlue}{rgb}{0.10,0.10,0.44}
\definecolor{MintCream}{rgb}{0.96,1.00,0.98}
\definecolor{MistyRose1}{rgb}{1.00,0.89,0.88}
\definecolor{MistyRose2}{rgb}{0.93,0.84,0.82}
\definecolor{MistyRose3}{rgb}{0.80,0.72,0.71}
\definecolor{MistyRose4}{rgb}{0.55,0.49,0.48}
\definecolor{MistyRose}{rgb}{1.00,0.89,0.88}
\definecolor{NavajoWhite1}{rgb}{1.00,0.87,0.68}
\definecolor{NavajoWhite2}{rgb}{0.93,0.81,0.63}
\definecolor{NavajoWhite3}{rgb}{0.80,0.70,0.55}
\definecolor{NavajoWhite4}{rgb}{0.55,0.47,0.37}
\definecolor{NavajoWhite}{rgb}{1.00,0.87,0.68}
\definecolor{NavyBlue}{rgb}{0.00,0.00,0.50}
\definecolor{OldLace}{rgb}{0.99,0.96,0.90}
\definecolor{OliveDrab1}{rgb}{0.75,1.00,0.24}
\definecolor{OliveDrab2}{rgb}{0.70,0.93,0.23}
\definecolor{OliveDrab3}{rgb}{0.60,0.80,0.20}
\definecolor{OliveDrab4}{rgb}{0.41,0.55,0.13}
\definecolor{OliveDrab}{rgb}{0.42,0.56,0.14}
\definecolor{OrangeRed1}{rgb}{1.00,0.27,0.00}
\definecolor{OrangeRed2}{rgb}{0.93,0.25,0.00}
\definecolor{OrangeRed3}{rgb}{0.80,0.22,0.00}
\definecolor{OrangeRed4}{rgb}{0.55,0.15,0.00}
\definecolor{OrangeRed}{rgb}{1.00,0.27,0.00}
\definecolor{PaleGoldenrod}{rgb}{0.93,0.91,0.67}
\definecolor{PaleGreen1}{rgb}{0.60,1.00,0.60}
\definecolor{PaleGreen2}{rgb}{0.56,0.93,0.56}
\definecolor{PaleGreen3}{rgb}{0.49,0.80,0.49}
\definecolor{PaleGreen4}{rgb}{0.33,0.55,0.33}
\definecolor{PaleGreen}{rgb}{0.60,0.98,0.60}
\definecolor{PaleTurquoise1}{rgb}{0.73,1.00,1.00}
\definecolor{PaleTurquoise2}{rgb}{0.68,0.93,0.93}
\definecolor{PaleTurquoise3}{rgb}{0.59,0.80,0.80}
\definecolor{PaleTurquoise4}{rgb}{0.40,0.55,0.55}
\definecolor{PaleTurquoise}{rgb}{0.69,0.93,0.93}
\definecolor{PaleVioletRed1}{rgb}{1.00,0.51,0.67}
\definecolor{PaleVioletRed2}{rgb}{0.93,0.47,0.62}
\definecolor{PaleVioletRed3}{rgb}{0.80,0.41,0.54}
\definecolor{PaleVioletRed4}{rgb}{0.55,0.28,0.36}
\definecolor{PaleVioletRed}{rgb}{0.86,0.44,0.58}
\definecolor{PapayaWhip}{rgb}{1.00,0.94,0.84}
\definecolor{PeachPuff1}{rgb}{1.00,0.85,0.73}
\definecolor{PeachPuff2}{rgb}{0.93,0.80,0.68}
\definecolor{PeachPuff3}{rgb}{0.80,0.69,0.58}
\definecolor{PeachPuff4}{rgb}{0.55,0.47,0.40}
\definecolor{PeachPuff}{rgb}{1.00,0.85,0.73}
\definecolor{PowderBlue}{rgb}{0.69,0.88,0.90}
\definecolor{RosyBrown1}{rgb}{1.00,0.76,0.76}
\definecolor{RosyBrown2}{rgb}{0.93,0.71,0.71}
\definecolor{RosyBrown3}{rgb}{0.80,0.61,0.61}
\definecolor{RosyBrown4}{rgb}{0.55,0.41,0.41}
\definecolor{RosyBrown}{rgb}{0.74,0.56,0.56}
\definecolor{RoyalBlue1}{rgb}{0.28,0.46,1.00}
\definecolor{RoyalBlue2}{rgb}{0.26,0.43,0.93}
\definecolor{RoyalBlue3}{rgb}{0.23,0.37,0.80}
\definecolor{RoyalBlue4}{rgb}{0.15,0.25,0.55}
\definecolor{RoyalBlue}{rgb}{0.25,0.41,0.88}
\definecolor{SaddleBrown}{rgb}{0.55,0.27,0.07}
\definecolor{SandyBrown}{rgb}{0.96,0.64,0.38}
\definecolor{SeaGreen1}{rgb}{0.33,1.00,0.62}
\definecolor{SeaGreen2}{rgb}{0.31,0.93,0.58}
\definecolor{SeaGreen3}{rgb}{0.26,0.80,0.50}
\definecolor{SeaGreen4}{rgb}{0.18,0.55,0.34}
\definecolor{SeaGreen}{rgb}{0.18,0.55,0.34}
\definecolor{SkyBlue1}{rgb}{0.53,0.81,1.00}
\definecolor{SkyBlue2}{rgb}{0.49,0.75,0.93}
\definecolor{SkyBlue3}{rgb}{0.42,0.65,0.80}
\definecolor{SkyBlue4}{rgb}{0.29,0.44,0.55}
\definecolor{SkyBlue}{rgb}{0.53,0.81,0.92}
\definecolor{SlateBlue1}{rgb}{0.51,0.44,1.00}
\definecolor{SlateBlue2}{rgb}{0.48,0.40,0.93}
\definecolor{SlateBlue3}{rgb}{0.41,0.35,0.80}
\definecolor{SlateBlue4}{rgb}{0.28,0.24,0.55}
\definecolor{SlateBlue}{rgb}{0.42,0.35,0.80}
\definecolor{SlateGray1}{rgb}{0.78,0.89,1.00}
\definecolor{SlateGray2}{rgb}{0.73,0.83,0.93}
\definecolor{SlateGray3}{rgb}{0.62,0.71,0.80}
\definecolor{SlateGray4}{rgb}{0.42,0.48,0.55}
\definecolor{SlateGray}{rgb}{0.44,0.50,0.56}
\definecolor{SlateGrey}{rgb}{0.44,0.50,0.56}
\definecolor{SpringGreen1}{rgb}{0.00,1.00,0.50}
\definecolor{SpringGreen2}{rgb}{0.00,0.93,0.46}
\definecolor{SpringGreen3}{rgb}{0.00,0.80,0.40}
\definecolor{SpringGreen4}{rgb}{0.00,0.55,0.27}
\definecolor{SpringGreen}{rgb}{0.00,1.00,0.50}
\definecolor{SteelBlue1}{rgb}{0.39,0.72,1.00}
\definecolor{SteelBlue2}{rgb}{0.36,0.67,0.93}
\definecolor{SteelBlue3}{rgb}{0.31,0.58,0.80}
\definecolor{SteelBlue4}{rgb}{0.21,0.39,0.55}
\definecolor{SteelBlue}{rgb}{0.27,0.51,0.71}
\definecolor{VioletRed1}{rgb}{1.00,0.24,0.59}
\definecolor{VioletRed2}{rgb}{0.93,0.23,0.55}
\definecolor{VioletRed3}{rgb}{0.80,0.20,0.47}
\definecolor{VioletRed4}{rgb}{0.55,0.13,0.32}
\definecolor{VioletRed}{rgb}{0.82,0.13,0.56}
\definecolor{WhiteSmoke}{rgb}{0.96,0.96,0.96}
\definecolor{YellowGreen}{rgb}{0.60,0.80,0.20}
\definecolor{aliceblue}{rgb}{0.94,0.97,1.00}
\definecolor{antiquewhite}{rgb}{0.98,0.92,0.84}
\definecolor{aquamarine1}{rgb}{0.50,1.00,0.83}
\definecolor{aquamarine2}{rgb}{0.46,0.93,0.78}
\definecolor{aquamarine3}{rgb}{0.40,0.80,0.67}
\definecolor{aquamarine4}{rgb}{0.27,0.55,0.45}
\definecolor{aquamarine}{rgb}{0.50,1.00,0.83}
\definecolor{azure1}{rgb}{0.94,1.00,1.00}
\definecolor{azure2}{rgb}{0.88,0.93,0.93}
\definecolor{azure3}{rgb}{0.76,0.80,0.80}
\definecolor{azure4}{rgb}{0.51,0.55,0.55}
\definecolor{azure}{rgb}{0.94,1.00,1.00}
\definecolor{beige}{rgb}{0.96,0.96,0.86}
\definecolor{bisque1}{rgb}{1.00,0.89,0.77}
\definecolor{bisque2}{rgb}{0.93,0.84,0.72}
\definecolor{bisque3}{rgb}{0.80,0.72,0.62}
\definecolor{bisque4}{rgb}{0.55,0.49,0.42}
\definecolor{bisque}{rgb}{1.00,0.89,0.77}
\definecolor{black}{rgb}{0.00,0.00,0.00}
\definecolor{blanchedalmond}{rgb}{1.00,0.92,0.80}
\definecolor{blue1}{rgb}{0.00,0.00,1.00}
\definecolor{blue2}{rgb}{0.00,0.00,0.93}
\definecolor{blue3}{rgb}{0.00,0.00,0.80}
\definecolor{blue4}{rgb}{0.00,0.00,0.55}
\definecolor{blueviolet}{rgb}{0.54,0.17,0.89}
\definecolor{blue}{rgb}{0.00,0.00,1.00}
\definecolor{brown1}{rgb}{1.00,0.25,0.25}
\definecolor{brown2}{rgb}{0.93,0.23,0.23}
\definecolor{brown3}{rgb}{0.80,0.20,0.20}
\definecolor{brown4}{rgb}{0.55,0.14,0.14}
\definecolor{brown}{rgb}{0.65,0.16,0.16}
\definecolor{burlywood1}{rgb}{1.00,0.83,0.61}
\definecolor{burlywood2}{rgb}{0.93,0.77,0.57}
\definecolor{burlywood3}{rgb}{0.80,0.67,0.49}
\definecolor{burlywood4}{rgb}{0.55,0.45,0.33}
\definecolor{burlywood}{rgb}{0.87,0.72,0.53}
\definecolor{cadetblue}{rgb}{0.37,0.62,0.63}
\definecolor{chartreuse1}{rgb}{0.50,1.00,0.00}
\definecolor{chartreuse2}{rgb}{0.46,0.93,0.00}
\definecolor{chartreuse3}{rgb}{0.40,0.80,0.00}
\definecolor{chartreuse4}{rgb}{0.27,0.55,0.00}
\definecolor{chartreuse}{rgb}{0.50,1.00,0.00}
\definecolor{chocolate1}{rgb}{1.00,0.50,0.14}
\definecolor{chocolate2}{rgb}{0.93,0.46,0.13}
\definecolor{chocolate3}{rgb}{0.80,0.40,0.11}
\definecolor{chocolate4}{rgb}{0.55,0.27,0.07}
\definecolor{chocolate}{rgb}{0.82,0.41,0.12}
\definecolor{coral1}{rgb}{1.00,0.45,0.34}
\definecolor{coral2}{rgb}{0.93,0.42,0.31}
\definecolor{coral3}{rgb}{0.80,0.36,0.27}
\definecolor{coral4}{rgb}{0.55,0.24,0.18}
\definecolor{coral}{rgb}{1.00,0.50,0.31}
\definecolor{cornflowerblue}{rgb}{0.39,0.58,0.93}
\definecolor{cornsilk1}{rgb}{1.00,0.97,0.86}
\definecolor{cornsilk2}{rgb}{0.93,0.91,0.80}
\definecolor{cornsilk3}{rgb}{0.80,0.78,0.69}
\definecolor{cornsilk4}{rgb}{0.55,0.53,0.47}
\definecolor{cornsilk}{rgb}{1.00,0.97,0.86}
\definecolor{cyan1}{rgb}{0.00,1.00,1.00}
\definecolor{cyan2}{rgb}{0.00,0.93,0.93}
\definecolor{cyan3}{rgb}{0.00,0.80,0.80}
\definecolor{cyan4}{rgb}{0.00,0.55,0.55}
\definecolor{cyan}{rgb}{0.00,1.00,1.00}
\definecolor{darkblue}{rgb}{0.00,0.00,0.55}
\definecolor{darkcyan}{rgb}{0.00,0.55,0.55}
\definecolor{darkgoldenrod}{rgb}{0.72,0.53,0.04}
\definecolor{darkgray}{rgb}{0.66,0.66,0.66}
\definecolor{darkgreen}{rgb}{0.00,0.39,0.00}
\definecolor{darkgrey}{rgb}{0.66,0.66,0.66}
\definecolor{darkkhaki}{rgb}{0.74,0.72,0.42}
\definecolor{darkmagenta}{rgb}{0.55,0.00,0.55}
\definecolor{darkolive}{rgb}{0.33,0.42,0.18}
\definecolor{darkorange}{rgb}{1.00,0.55,0.00}
\definecolor{darkorchid}{rgb}{0.60,0.20,0.80}
\definecolor{darkred}{rgb}{0.55,0.00,0.00}
\definecolor{darksalmon}{rgb}{0.91,0.59,0.48}
\definecolor{darksea}{rgb}{0.56,0.74,0.56}
\definecolor{darkslate}{rgb}{0.18,0.31,0.31}
\definecolor{darkslate}{rgb}{0.18,0.31,0.31}
\definecolor{darkslate}{rgb}{0.28,0.24,0.55}
\definecolor{darkturquoise}{rgb}{0.00,0.81,0.82}
\definecolor{darkviolet}{rgb}{0.58,0.00,0.83}
\definecolor{deeppink}{rgb}{1.00,0.08,0.58}
\definecolor{deepsky}{rgb}{0.00,0.75,1.00}
\definecolor{dimgray}{rgb}{0.41,0.41,0.41}
\definecolor{dimgrey}{rgb}{0.41,0.41,0.41}
\definecolor{dodgerblue}{rgb}{0.12,0.56,1.00}
\definecolor{firebrick1}{rgb}{1.00,0.19,0.19}
\definecolor{firebrick2}{rgb}{0.93,0.17,0.17}
\definecolor{firebrick3}{rgb}{0.80,0.15,0.15}
\definecolor{firebrick4}{rgb}{0.55,0.10,0.10}
\definecolor{firebrick}{rgb}{0.70,0.13,0.13}
\definecolor{floralwhite}{rgb}{1.00,0.98,0.94}
\definecolor{forestgreen}{rgb}{0.13,0.55,0.13}
\definecolor{gainsboro}{rgb}{0.86,0.86,0.86}
\definecolor{ghostwhite}{rgb}{0.97,0.97,1.00}
\definecolor{gold1}{rgb}{1.00,0.84,0.00}
\definecolor{gold2}{rgb}{0.93,0.79,0.00}
\definecolor{gold3}{rgb}{0.80,0.68,0.00}
\definecolor{gold4}{rgb}{0.55,0.46,0.00}
\definecolor{goldenrod1}{rgb}{1.00,0.76,0.15}
\definecolor{goldenrod2}{rgb}{0.93,0.71,0.13}
\definecolor{goldenrod3}{rgb}{0.80,0.61,0.11}
\definecolor{goldenrod4}{rgb}{0.55,0.41,0.08}
\definecolor{goldenrod}{rgb}{0.85,0.65,0.13}
\definecolor{gold}{rgb}{1.00,0.84,0.00}
\definecolor{gray0}{rgb}{0.00,0.00,0.00}
\definecolor{gray100}{rgb}{1.00,1.00,1.00}
\definecolor{gray10}{rgb}{0.10,0.10,0.10}
\definecolor{gray11}{rgb}{0.11,0.11,0.11}
\definecolor{gray12}{rgb}{0.12,0.12,0.12}
\definecolor{gray13}{rgb}{0.13,0.13,0.13}
\definecolor{gray14}{rgb}{0.14,0.14,0.14}
\definecolor{gray15}{rgb}{0.15,0.15,0.15}
\definecolor{gray16}{rgb}{0.16,0.16,0.16}
\definecolor{gray17}{rgb}{0.17,0.17,0.17}
\definecolor{gray18}{rgb}{0.18,0.18,0.18}
\definecolor{gray19}{rgb}{0.19,0.19,0.19}
\definecolor{gray1}{rgb}{0.01,0.01,0.01}
\definecolor{gray20}{rgb}{0.20,0.20,0.20}
\definecolor{gray21}{rgb}{0.21,0.21,0.21}
\definecolor{gray22}{rgb}{0.22,0.22,0.22}
\definecolor{gray23}{rgb}{0.23,0.23,0.23}
\definecolor{gray24}{rgb}{0.24,0.24,0.24}
\definecolor{gray25}{rgb}{0.25,0.25,0.25}
\definecolor{gray26}{rgb}{0.26,0.26,0.26}
\definecolor{gray27}{rgb}{0.27,0.27,0.27}
\definecolor{gray28}{rgb}{0.28,0.28,0.28}
\definecolor{gray29}{rgb}{0.29,0.29,0.29}
\definecolor{gray2}{rgb}{0.02,0.02,0.02}
\definecolor{gray30}{rgb}{0.30,0.30,0.30}
\definecolor{gray31}{rgb}{0.31,0.31,0.31}
\definecolor{gray32}{rgb}{0.32,0.32,0.32}
\definecolor{gray33}{rgb}{0.33,0.33,0.33}
\definecolor{gray34}{rgb}{0.34,0.34,0.34}
\definecolor{gray35}{rgb}{0.35,0.35,0.35}
\definecolor{gray36}{rgb}{0.36,0.36,0.36}
\definecolor{gray37}{rgb}{0.37,0.37,0.37}
\definecolor{gray38}{rgb}{0.38,0.38,0.38}
\definecolor{gray39}{rgb}{0.39,0.39,0.39}
\definecolor{gray3}{rgb}{0.03,0.03,0.03}
\definecolor{gray40}{rgb}{0.40,0.40,0.40}
\definecolor{gray41}{rgb}{0.41,0.41,0.41}
\definecolor{gray42}{rgb}{0.42,0.42,0.42}
\definecolor{gray43}{rgb}{0.43,0.43,0.43}
\definecolor{gray44}{rgb}{0.44,0.44,0.44}
\definecolor{gray45}{rgb}{0.45,0.45,0.45}
\definecolor{gray46}{rgb}{0.46,0.46,0.46}
\definecolor{gray47}{rgb}{0.47,0.47,0.47}
\definecolor{gray48}{rgb}{0.48,0.48,0.48}
\definecolor{gray49}{rgb}{0.49,0.49,0.49}
\definecolor{gray4}{rgb}{0.04,0.04,0.04}
\definecolor{gray50}{rgb}{0.50,0.50,0.50}
\definecolor{gray51}{rgb}{0.51,0.51,0.51}
\definecolor{gray52}{rgb}{0.52,0.52,0.52}
\definecolor{gray53}{rgb}{0.53,0.53,0.53}
\definecolor{gray54}{rgb}{0.54,0.54,0.54}
\definecolor{gray55}{rgb}{0.55,0.55,0.55}
\definecolor{gray56}{rgb}{0.56,0.56,0.56}
\definecolor{gray57}{rgb}{0.57,0.57,0.57}
\definecolor{gray58}{rgb}{0.58,0.58,0.58}
\definecolor{gray59}{rgb}{0.59,0.59,0.59}
\definecolor{gray5}{rgb}{0.05,0.05,0.05}
\definecolor{gray60}{rgb}{0.60,0.60,0.60}
\definecolor{gray61}{rgb}{0.61,0.61,0.61}
\definecolor{gray62}{rgb}{0.62,0.62,0.62}
\definecolor{gray63}{rgb}{0.63,0.63,0.63}
\definecolor{gray64}{rgb}{0.64,0.64,0.64}
\definecolor{gray65}{rgb}{0.65,0.65,0.65}
\definecolor{gray66}{rgb}{0.66,0.66,0.66}
\definecolor{gray67}{rgb}{0.67,0.67,0.67}
\definecolor{gray68}{rgb}{0.68,0.68,0.68}
\definecolor{gray69}{rgb}{0.69,0.69,0.69}
\definecolor{gray6}{rgb}{0.06,0.06,0.06}
\definecolor{gray70}{rgb}{0.70,0.70,0.70}
\definecolor{gray71}{rgb}{0.71,0.71,0.71}
\definecolor{gray72}{rgb}{0.72,0.72,0.72}
\definecolor{gray73}{rgb}{0.73,0.73,0.73}
\definecolor{gray74}{rgb}{0.74,0.74,0.74}
\definecolor{gray75}{rgb}{0.75,0.75,0.75}
\definecolor{gray76}{rgb}{0.76,0.76,0.76}
\definecolor{gray77}{rgb}{0.77,0.77,0.77}
\definecolor{gray78}{rgb}{0.78,0.78,0.78}
\definecolor{gray79}{rgb}{0.79,0.79,0.79}
\definecolor{gray7}{rgb}{0.07,0.07,0.07}
\definecolor{gray80}{rgb}{0.80,0.80,0.80}
\definecolor{gray81}{rgb}{0.81,0.81,0.81}
\definecolor{gray82}{rgb}{0.82,0.82,0.82}
\definecolor{gray83}{rgb}{0.83,0.83,0.83}
\definecolor{gray84}{rgb}{0.84,0.84,0.84}
\definecolor{gray85}{rgb}{0.85,0.85,0.85}
\definecolor{gray86}{rgb}{0.86,0.86,0.86}
\definecolor{gray87}{rgb}{0.87,0.87,0.87}
\definecolor{gray88}{rgb}{0.88,0.88,0.88}
\definecolor{gray89}{rgb}{0.89,0.89,0.89}
\definecolor{gray8}{rgb}{0.08,0.08,0.08}
\definecolor{gray90}{rgb}{0.90,0.90,0.90}
\definecolor{gray91}{rgb}{0.91,0.91,0.91}
\definecolor{gray92}{rgb}{0.92,0.92,0.92}
\definecolor{gray93}{rgb}{0.93,0.93,0.93}
\definecolor{gray94}{rgb}{0.94,0.94,0.94}
\definecolor{gray95}{rgb}{0.95,0.95,0.95}
\definecolor{gray96}{rgb}{0.96,0.96,0.96}
\definecolor{gray97}{rgb}{0.97,0.97,0.97}
\definecolor{gray98}{rgb}{0.98,0.98,0.98}
\definecolor{gray99}{rgb}{0.99,0.99,0.99}
\definecolor{gray9}{rgb}{0.09,0.09,0.09}
\definecolor{gray}{rgb}{0.75,0.75,0.75}
\definecolor{green1}{rgb}{0.00,1.00,0.00}
\definecolor{green2}{rgb}{0.00,0.93,0.00}
\definecolor{green3}{rgb}{0.00,0.80,0.00}
\definecolor{green4}{rgb}{0.00,0.55,0.00}
\definecolor{greenyellow}{rgb}{0.68,1.00,0.18}
\definecolor{green}{rgb}{0.00,1.00,0.00}
\definecolor{grey0}{rgb}{0.00,0.00,0.00}
\definecolor{grey100}{rgb}{1.00,1.00,1.00}
\definecolor{grey10}{rgb}{0.10,0.10,0.10}
\definecolor{grey11}{rgb}{0.11,0.11,0.11}
\definecolor{grey12}{rgb}{0.12,0.12,0.12}
\definecolor{grey13}{rgb}{0.13,0.13,0.13}
\definecolor{grey14}{rgb}{0.14,0.14,0.14}
\definecolor{grey15}{rgb}{0.15,0.15,0.15}
\definecolor{grey16}{rgb}{0.16,0.16,0.16}
\definecolor{grey17}{rgb}{0.17,0.17,0.17}
\definecolor{grey18}{rgb}{0.18,0.18,0.18}
\definecolor{grey19}{rgb}{0.19,0.19,0.19}
\definecolor{grey1}{rgb}{0.01,0.01,0.01}
\definecolor{grey20}{rgb}{0.20,0.20,0.20}
\definecolor{grey21}{rgb}{0.21,0.21,0.21}
\definecolor{grey22}{rgb}{0.22,0.22,0.22}
\definecolor{grey23}{rgb}{0.23,0.23,0.23}
\definecolor{grey24}{rgb}{0.24,0.24,0.24}
\definecolor{grey25}{rgb}{0.25,0.25,0.25}
\definecolor{grey26}{rgb}{0.26,0.26,0.26}
\definecolor{grey27}{rgb}{0.27,0.27,0.27}
\definecolor{grey28}{rgb}{0.28,0.28,0.28}
\definecolor{grey29}{rgb}{0.29,0.29,0.29}
\definecolor{grey2}{rgb}{0.02,0.02,0.02}
\definecolor{grey30}{rgb}{0.30,0.30,0.30}
\definecolor{grey31}{rgb}{0.31,0.31,0.31}
\definecolor{grey32}{rgb}{0.32,0.32,0.32}
\definecolor{grey33}{rgb}{0.33,0.33,0.33}
\definecolor{grey34}{rgb}{0.34,0.34,0.34}
\definecolor{grey35}{rgb}{0.35,0.35,0.35}
\definecolor{grey36}{rgb}{0.36,0.36,0.36}
\definecolor{grey37}{rgb}{0.37,0.37,0.37}
\definecolor{grey38}{rgb}{0.38,0.38,0.38}
\definecolor{grey39}{rgb}{0.39,0.39,0.39}
\definecolor{grey3}{rgb}{0.03,0.03,0.03}
\definecolor{grey40}{rgb}{0.40,0.40,0.40}
\definecolor{grey41}{rgb}{0.41,0.41,0.41}
\definecolor{grey42}{rgb}{0.42,0.42,0.42}
\definecolor{grey43}{rgb}{0.43,0.43,0.43}
\definecolor{grey44}{rgb}{0.44,0.44,0.44}
\definecolor{grey45}{rgb}{0.45,0.45,0.45}
\definecolor{grey46}{rgb}{0.46,0.46,0.46}
\definecolor{grey47}{rgb}{0.47,0.47,0.47}
\definecolor{grey48}{rgb}{0.48,0.48,0.48}
\definecolor{grey49}{rgb}{0.49,0.49,0.49}
\definecolor{grey4}{rgb}{0.04,0.04,0.04}
\definecolor{grey50}{rgb}{0.50,0.50,0.50}
\definecolor{grey51}{rgb}{0.51,0.51,0.51}
\definecolor{grey52}{rgb}{0.52,0.52,0.52}
\definecolor{grey53}{rgb}{0.53,0.53,0.53}
\definecolor{grey54}{rgb}{0.54,0.54,0.54}
\definecolor{grey55}{rgb}{0.55,0.55,0.55}
\definecolor{grey56}{rgb}{0.56,0.56,0.56}
\definecolor{grey57}{rgb}{0.57,0.57,0.57}
\definecolor{grey58}{rgb}{0.58,0.58,0.58}
\definecolor{grey59}{rgb}{0.59,0.59,0.59}
\definecolor{grey5}{rgb}{0.05,0.05,0.05}
\definecolor{grey60}{rgb}{0.60,0.60,0.60}
\definecolor{grey61}{rgb}{0.61,0.61,0.61}
\definecolor{grey62}{rgb}{0.62,0.62,0.62}
\definecolor{grey63}{rgb}{0.63,0.63,0.63}
\definecolor{grey64}{rgb}{0.64,0.64,0.64}
\definecolor{grey65}{rgb}{0.65,0.65,0.65}
\definecolor{grey66}{rgb}{0.66,0.66,0.66}
\definecolor{grey67}{rgb}{0.67,0.67,0.67}
\definecolor{grey68}{rgb}{0.68,0.68,0.68}
\definecolor{grey69}{rgb}{0.69,0.69,0.69}
\definecolor{grey6}{rgb}{0.06,0.06,0.06}
\definecolor{grey70}{rgb}{0.70,0.70,0.70}
\definecolor{grey71}{rgb}{0.71,0.71,0.71}
\definecolor{grey72}{rgb}{0.72,0.72,0.72}
\definecolor{grey73}{rgb}{0.73,0.73,0.73}
\definecolor{grey74}{rgb}{0.74,0.74,0.74}
\definecolor{grey75}{rgb}{0.75,0.75,0.75}
\definecolor{grey76}{rgb}{0.76,0.76,0.76}
\definecolor{grey77}{rgb}{0.77,0.77,0.77}
\definecolor{grey78}{rgb}{0.78,0.78,0.78}
\definecolor{grey79}{rgb}{0.79,0.79,0.79}
\definecolor{grey7}{rgb}{0.07,0.07,0.07}
\definecolor{grey80}{rgb}{0.80,0.80,0.80}
\definecolor{grey81}{rgb}{0.81,0.81,0.81}
\definecolor{grey82}{rgb}{0.82,0.82,0.82}
\definecolor{grey83}{rgb}{0.83,0.83,0.83}
\definecolor{grey84}{rgb}{0.84,0.84,0.84}
\definecolor{grey85}{rgb}{0.85,0.85,0.85}
\definecolor{grey86}{rgb}{0.86,0.86,0.86}
\definecolor{grey87}{rgb}{0.87,0.87,0.87}
\definecolor{grey88}{rgb}{0.88,0.88,0.88}
\definecolor{grey89}{rgb}{0.89,0.89,0.89}
\definecolor{grey8}{rgb}{0.08,0.08,0.08}
\definecolor{grey90}{rgb}{0.90,0.90,0.90}
\definecolor{grey91}{rgb}{0.91,0.91,0.91}
\definecolor{grey92}{rgb}{0.92,0.92,0.92}
\definecolor{grey93}{rgb}{0.93,0.93,0.93}
\definecolor{grey94}{rgb}{0.94,0.94,0.94}
\definecolor{grey95}{rgb}{0.95,0.95,0.95}
\definecolor{grey96}{rgb}{0.96,0.96,0.96}
\definecolor{grey97}{rgb}{0.97,0.97,0.97}
\definecolor{grey98}{rgb}{0.98,0.98,0.98}
\definecolor{grey99}{rgb}{0.99,0.99,0.99}
\definecolor{grey9}{rgb}{0.09,0.09,0.09}
\definecolor{grey}{rgb}{0.75,0.75,0.75}
\definecolor{honeydew1}{rgb}{0.94,1.00,0.94}
\definecolor{honeydew2}{rgb}{0.88,0.93,0.88}
\definecolor{honeydew3}{rgb}{0.76,0.80,0.76}
\definecolor{honeydew4}{rgb}{0.51,0.55,0.51}
\definecolor{honeydew}{rgb}{0.94,1.00,0.94}
\definecolor{hotpink}{rgb}{1.00,0.41,0.71}
\definecolor{indianred}{rgb}{0.80,0.36,0.36}
\definecolor{ivory1}{rgb}{1.00,1.00,0.94}
\definecolor{ivory2}{rgb}{0.93,0.93,0.88}
\definecolor{ivory3}{rgb}{0.80,0.80,0.76}
\definecolor{ivory4}{rgb}{0.55,0.55,0.51}
\definecolor{ivory}{rgb}{1.00,1.00,0.94}
\definecolor{khaki1}{rgb}{1.00,0.96,0.56}
\definecolor{khaki2}{rgb}{0.93,0.90,0.52}
\definecolor{khaki3}{rgb}{0.80,0.78,0.45}
\definecolor{khaki4}{rgb}{0.55,0.53,0.31}
\definecolor{khaki}{rgb}{0.94,0.90,0.55}
\definecolor{lavenderblush}{rgb}{1.00,0.94,0.96}
\definecolor{lavender}{rgb}{0.90,0.90,0.98}
\definecolor{lawngreen}{rgb}{0.49,0.99,0.00}
\definecolor{lemonchiffon}{rgb}{1.00,0.98,0.80}
\definecolor{lightblue}{rgb}{0.68,0.85,0.90}
\definecolor{lightcoral}{rgb}{0.94,0.50,0.50}
\definecolor{lightcyan}{rgb}{0.88,1.00,1.00}
\definecolor{lightgoldenrod}{rgb}{0.93,0.87,0.51}
\definecolor{lightgoldenrod}{rgb}{0.98,0.98,0.82}
\definecolor{lightgray}{rgb}{0.83,0.83,0.83}
\definecolor{lightgreen}{rgb}{0.56,0.93,0.56}
\definecolor{lightgrey}{rgb}{0.83,0.83,0.83}
\definecolor{lightpink}{rgb}{1.00,0.71,0.76}
\definecolor{lightsalmon}{rgb}{1.00,0.63,0.48}
\definecolor{lightsea}{rgb}{0.13,0.70,0.67}
\definecolor{lightsky}{rgb}{0.53,0.81,0.98}
\definecolor{lightslate}{rgb}{0.47,0.53,0.60}
\definecolor{lightslate}{rgb}{0.47,0.53,0.60}
\definecolor{lightslate}{rgb}{0.52,0.44,1.00}
\definecolor{lightsteel}{rgb}{0.69,0.77,0.87}
\definecolor{lightyellow}{rgb}{1.00,1.00,0.88}
\definecolor{limegreen}{rgb}{0.20,0.80,0.20}
\definecolor{linen}{rgb}{0.98,0.94,0.90}
\definecolor{magenta1}{rgb}{1.00,0.00,1.00}
\definecolor{magenta2}{rgb}{0.93,0.00,0.93}
\definecolor{magenta3}{rgb}{0.80,0.00,0.80}
\definecolor{magenta4}{rgb}{0.55,0.00,0.55}
\definecolor{magenta}{rgb}{1.00,0.00,1.00}
\definecolor{maroon1}{rgb}{1.00,0.20,0.70}
\definecolor{maroon2}{rgb}{0.93,0.19,0.65}
\definecolor{maroon3}{rgb}{0.80,0.16,0.56}
\definecolor{maroon4}{rgb}{0.55,0.11,0.38}
\definecolor{maroon}{rgb}{0.69,0.19,0.38}
\definecolor{mediumaquamarine}{rgb}{0.40,0.80,0.67}
\definecolor{mediumblue}{rgb}{0.00,0.00,0.80}
\definecolor{mediumorchid}{rgb}{0.73,0.33,0.83}
\definecolor{mediumpurple}{rgb}{0.58,0.44,0.86}
\definecolor{mediumsea}{rgb}{0.24,0.70,0.44}
\definecolor{mediumslate}{rgb}{0.48,0.41,0.93}
\definecolor{mediumspring}{rgb}{0.00,0.98,0.60}
\definecolor{mediumturquoise}{rgb}{0.28,0.82,0.80}
\definecolor{mediumviolet}{rgb}{0.78,0.08,0.52}
\definecolor{midnightblue}{rgb}{0.10,0.10,0.44}
\definecolor{mintcream}{rgb}{0.96,1.00,0.98}
\definecolor{mistyrose}{rgb}{1.00,0.89,0.88}
\definecolor{moccasin}{rgb}{1.00,0.89,0.71}
\definecolor{navajowhite}{rgb}{1.00,0.87,0.68}
\definecolor{navyblue}{rgb}{0.00,0.00,0.50}
\definecolor{navy}{rgb}{0.00,0.00,0.50}
\definecolor{oldlace}{rgb}{0.99,0.96,0.90}
\definecolor{olivedrab}{rgb}{0.42,0.56,0.14}
\definecolor{orange1}{rgb}{1.00,0.65,0.00}
\definecolor{orange2}{rgb}{0.93,0.60,0.00}
\definecolor{orange3}{rgb}{0.80,0.52,0.00}
\definecolor{orange4}{rgb}{0.55,0.35,0.00}
\definecolor{orangered}{rgb}{1.00,0.27,0.00}
\definecolor{orange}{rgb}{1.00,0.65,0.00}
\definecolor{orchid1}{rgb}{1.00,0.51,0.98}
\definecolor{orchid2}{rgb}{0.93,0.48,0.91}
\definecolor{orchid3}{rgb}{0.80,0.41,0.79}
\definecolor{orchid4}{rgb}{0.55,0.28,0.54}
\definecolor{orchid}{rgb}{0.85,0.44,0.84}
\definecolor{palegoldenrod}{rgb}{0.93,0.91,0.67}
\definecolor{palegreen}{rgb}{0.60,0.98,0.60}
\definecolor{paleturquoise}{rgb}{0.69,0.93,0.93}
\definecolor{paleviolet}{rgb}{0.86,0.44,0.58}
\definecolor{papayawhip}{rgb}{1.00,0.94,0.84}
\definecolor{peachpuff}{rgb}{1.00,0.85,0.73}
\definecolor{peru}{rgb}{0.80,0.52,0.25}
\definecolor{pink1}{rgb}{1.00,0.71,0.77}
\definecolor{pink2}{rgb}{0.93,0.66,0.72}
\definecolor{pink3}{rgb}{0.80,0.57,0.62}
\definecolor{pink4}{rgb}{0.55,0.39,0.42}
\definecolor{pink}{rgb}{1.00,0.75,0.80}
\definecolor{plum1}{rgb}{1.00,0.73,1.00}
\definecolor{plum2}{rgb}{0.93,0.68,0.93}
\definecolor{plum3}{rgb}{0.80,0.59,0.80}
\definecolor{plum4}{rgb}{0.55,0.40,0.55}
\definecolor{plum}{rgb}{0.87,0.63,0.87}
\definecolor{powderblue}{rgb}{0.69,0.88,0.90}
\definecolor{purple1}{rgb}{0.61,0.19,1.00}
\definecolor{purple2}{rgb}{0.57,0.17,0.93}
\definecolor{purple3}{rgb}{0.49,0.15,0.80}
\definecolor{purple4}{rgb}{0.33,0.10,0.55}
\definecolor{purple}{rgb}{0.63,0.13,0.94}
\definecolor{red1}{rgb}{1.00,0.00,0.00}
\definecolor{red2}{rgb}{0.93,0.00,0.00}
\definecolor{red3}{rgb}{0.80,0.00,0.00}
\definecolor{red4}{rgb}{0.55,0.00,0.00}
\definecolor{red}{rgb}{1.00,0.00,0.00}
\definecolor{rosybrown}{rgb}{0.74,0.56,0.56}
\definecolor{royalblue}{rgb}{0.25,0.41,0.88}
\definecolor{saddlebrown}{rgb}{0.55,0.27,0.07}
\definecolor{salmon1}{rgb}{1.00,0.55,0.41}
\definecolor{salmon2}{rgb}{0.93,0.51,0.38}
\definecolor{salmon3}{rgb}{0.80,0.44,0.33}
\definecolor{salmon4}{rgb}{0.55,0.30,0.22}
\definecolor{salmon}{rgb}{0.98,0.50,0.45}
\definecolor{sandybrown}{rgb}{0.96,0.64,0.38}
\definecolor{seagreen}{rgb}{0.18,0.55,0.34}
\definecolor{seashell1}{rgb}{1.00,0.96,0.93}
\definecolor{seashell2}{rgb}{0.93,0.90,0.87}
\definecolor{seashell3}{rgb}{0.80,0.77,0.75}
\definecolor{seashell4}{rgb}{0.55,0.53,0.51}
\definecolor{seashell}{rgb}{1.00,0.96,0.93}
\definecolor{sienna1}{rgb}{1.00,0.51,0.28}
\definecolor{sienna2}{rgb}{0.93,0.47,0.26}
\definecolor{sienna3}{rgb}{0.80,0.41,0.22}
\definecolor{sienna4}{rgb}{0.55,0.28,0.15}
\definecolor{sienna}{rgb}{0.63,0.32,0.18}
\definecolor{skyblue}{rgb}{0.53,0.81,0.92}
\definecolor{slateblue}{rgb}{0.42,0.35,0.80}
\definecolor{slategray}{rgb}{0.44,0.50,0.56}
\definecolor{slategrey}{rgb}{0.44,0.50,0.56}
\definecolor{snow1}{rgb}{1.00,0.98,0.98}
\definecolor{snow2}{rgb}{0.93,0.91,0.91}
\definecolor{snow3}{rgb}{0.80,0.79,0.79}
\definecolor{snow4}{rgb}{0.55,0.54,0.54}
\definecolor{snow}{rgb}{1.00,0.98,0.98}
\definecolor{springgreen}{rgb}{0.00,1.00,0.50}
\definecolor{steelblue}{rgb}{0.27,0.51,0.71}
\definecolor{tan1}{rgb}{1.00,0.65,0.31}
\definecolor{tan2}{rgb}{0.93,0.60,0.29}
\definecolor{tan3}{rgb}{0.80,0.52,0.25}
\definecolor{tan4}{rgb}{0.55,0.35,0.17}
\definecolor{tan}{rgb}{0.82,0.71,0.55}
\definecolor{thistle1}{rgb}{1.00,0.88,1.00}
\definecolor{thistle2}{rgb}{0.93,0.82,0.93}
\definecolor{thistle3}{rgb}{0.80,0.71,0.80}
\definecolor{thistle4}{rgb}{0.55,0.48,0.55}
\definecolor{thistle}{rgb}{0.85,0.75,0.85}
\definecolor{tomato1}{rgb}{1.00,0.39,0.28}
\definecolor{tomato2}{rgb}{0.93,0.36,0.26}
\definecolor{tomato3}{rgb}{0.80,0.31,0.22}
\definecolor{tomato4}{rgb}{0.55,0.21,0.15}
\definecolor{tomato}{rgb}{1.00,0.39,0.28}
\definecolor{turquoise1}{rgb}{0.00,0.96,1.00}
\definecolor{turquoise2}{rgb}{0.00,0.90,0.93}
\definecolor{turquoise3}{rgb}{0.00,0.77,0.80}
\definecolor{turquoise4}{rgb}{0.00,0.53,0.55}
\definecolor{turquoise}{rgb}{0.25,0.88,0.82}
\definecolor{violetred}{rgb}{0.82,0.13,0.56}
\definecolor{violet}{rgb}{0.93,0.51,0.93}
\definecolor{wheat1}{rgb}{1.00,0.91,0.73}
\definecolor{wheat2}{rgb}{0.93,0.85,0.68}
\definecolor{wheat3}{rgb}{0.80,0.73,0.59}
\definecolor{wheat4}{rgb}{0.55,0.49,0.40}
\definecolor{wheat}{rgb}{0.96,0.87,0.70}
\definecolor{whitesmoke}{rgb}{0.96,0.96,0.96}
\definecolor{white}{rgb}{1.00,1.00,1.00}
\definecolor{yellow1}{rgb}{1.00,1.00,0.00}
\definecolor{yellow2}{rgb}{0.93,0.93,0.00}
\definecolor{yellow3}{rgb}{0.80,0.80,0.00}
\definecolor{yellow4}{rgb}{0.55,0.55,0.00}
\definecolor{yellowgreen}{rgb}{0.60,0.80,0.20}
\definecolor{yellow}{rgb}{1.00,1.00,0.00}


\oddsidemargin -0.5in
\evensidemargin -0.5in
\topmargin -0.5in
\textwidth 7.5in
\textheight 9in

\newcommand{\EasyBMP}{\textbf{\texttt{{\color{DarkBlue}Easy}{\color{DarkRed}BMP}}}}
\newcommand{\ExampleBox}[1]{\begin{center}\framebox{\parbox[t]{6in}{\underline{Example:}\\ \\ #1}}\end{center}}
\newcommand{\PreviousVersion}{1.03}
\newcommand{\Version}{1.06}

\newcommand{\EmailAddress}{\texttt{macklin01@users.sourceforge.net}}
\newcommand{\WebAddress}{\texttt{http://easybmp.sourceforge.net}}

\begin{document}
\pagestyle{fancyplain}
% \markboth{}{\EasyBMP{} User Manual (Version \Version{})}
\lhead[]{\EasyBMP{} User Manual}
\rhead[]{Version \Version{}}
\cfoot[\thepage]{Page \thepage}

\title{\EasyBMP{} User Manual (Version \Version{})}
\author{Paul Macklin\\
\\
\textbf{email:} \EmailAddress{}\\
\textbf{WWW:} \WebAddress{}}
\date{\today}
\maketitle

\abstract{We define and document a simple, easy-to-use, 
cross-platform/cross-architecture Windows bitmap (BMP) library 
written in C++.  The \EasyBMP{} library will work for 
input and output on uncompressed 1, 
4, 8, 16, 24, and 32 bpp (bits per pixel) Windows BMP files in just about 
any operating system on just about any 32-bit or higher architecture. 
\EasyBMP{} has been tested on both little-endian (Pentium 3, Pentium 4, 
Celeron, Celeron M, Pentium M) and big-endian (Sun Sparc4) machines 
in Linux, Solaris, and Windows on 32-bit and 64-bit CPU's. 
Currently, GNU g++ (along with 
variants and/or frontends such as MinGW and Bloodshed), Intel's compiler, 
Borland's compiler, and Microsoft's MSVC++ compilers are 
fully supported.\\

\EasyBMP{} is open source software and is licensed under the 
revised/modifed BSD license. Please see the \texttt{BSD\_(revised)\_license.txt} 
file for further details. If you use this library in your application, it is the 
author's request that you notify him.
} 

\tableofcontents

\section{What's New in this Release (Version \Version{})}
Several major changes have been made since Version \PreviousVersion{}, 
including bugfixes, improved robustness, code cleanups, 
and feature enhancements. 

\subsection{Changes in Version 1.04}
Version 1.04 focused on improving compatibility with the Borland 
compiler. Very few warnings are generated when compiling; 
those that remain can be safely ignored and point to quirks 
of Borland rather than the code. \\

This release also added some new functionality: the ability 
to suppress all EasyBMP warning and error messages. This new 
feature should be particularly useful for projects that have 
no need for terminal output. 

\subsection{Changes in Version 1.05}
Version 1.05 improved compatibility by adding a copy constructor 
for the BMP class. A new feature was also added: bilinear image 
rescaling to a desired percentage, width, height, or square 
box fit. 

\subsection{Changes in Version 1.06}
Version 1.06 fixed several minor bugs, particularly in the 
copy constructor. (Thank you to ``redmaya'' in South Korea!) 
The \texttt{cctype} and \texttt{cstring} includes were added 
to improve compliance and compatibility. Lastly, 
\texttt{GetPixel()} and \texttt{SetPixel()} functions were 
added for future use, where we hope to be more careful with the 
\texttt{const} keyword. \\

\hrule\vskip 0.01in\hrule\vskip 0.25in

For further information on the changes made in \EasyBMP{}, 
please see the \texttt{EasyBMP\_ChangeLog.txt} file that is 
included with every release. 

\section{Introduction to the \EasyBMP{} Library}
During my studies at the University of Minnesota 
and the University of California, I needed a simple 
method to create and modify images. Because the Windows BMP file format 
is nearly universally readable, flexible, and simple, I decided to 
work with this format. (No compression to worry about, potential 
for 8 bits per color channel, etc.) \\

There are many excellent open- and closed-source BMP 
and image libraries available, and in no way do I claim that 
anything here is even equal to those libraries. 
However, as I looked about I noticed that quite a few
existing libraries had one or more of the following properties:
\begin{list}{$\bullet\hspace{.125in}$}{}
\item
too feature-rich (and accordingly more difficult to learn);
\item
required extensive installation;
\item
relied upon Linux or Windows libraries for simple functions;
\item
were too poorly documented for the novice programmer;
\item
required programming changes when moving code from one platform to another.
\end{list}
At that point, I decided to create my \EasyBMP{} library. 
My goals included easy inclusion in C++ projects, ease of 
use, no dependence upon other libraries 
(totally self-contained), and cross-platform compatibility.

\subsection{Sample Application: Converting a Color 
Image to Grayscale}
Here, we give a first sample application using the \EasyBMP{} library. 
Notice that inclusion of the library is simple: we include 
the \texttt{EasyBMP.h} file.  In this application, we see a simple 
example of opening an existing BMP file, reading its RGB information, 
and manipulating and writing that information to another BMP file. 
The commands are pretty straightforward. This example should illustrate 
how easy the library is for even the novice programmer.

\begin{verbatim}
#include "EasyBMP.h"
using namespace std;

int main( int argc, char* argv[] )
{
 if( argc != 3 )
 {
  cout << "Usage: ColorBMPtoGrayscale <input_filename> <output_filename>" 
       << endl << endl;
  return 1; 
 }  

 // declare and read the bitmap
 BMP Input;
 Input.ReadFromFile( argv[1] );

 // convert each pixel to grayscale using RGB->YUV
 for( int j=0 ; j < Input.TellHeight() ; j++)
 {
  for( int i=0 ; i < Input.TellWidth() ; i++)
  {
   int Temp = (int) floor( 0.299*Input(i,j)->Red + 
                           0.587*Input(i,j)->Green + 
                           0.114*Input(i,j)->Blue );
   ebmpBYTE TempBYTE = (ebmpBYTE) Temp;
   Input(i,j)->Red   = TempBYTE;
   Input(i,j)->Green = TempBYTE;
   Input(i,j)->Blue  = TempBYTE;
  }
 }
 
 // Create a grayscale color table if necessary 
 if( Input.TellBitDepth() < 16 )
 { CreateGrayscaleColorTable( Input ); }

 // write the output file
 Input.WriteToFile( argv[2] );

 return 0;
}
\end{verbatim}
Additional code samples are available for download at \\

\WebAddress{}

\section{What's \EasyBMP{} Good for?}
Lots of things! Okay, so we're a little biased. :-) \EasyBMP{} was first 
used to easily load textures in a homebrew raytracer. Later on, however, 
its focus shifted to being a simple, intuitive way to get image 
data in and out of programs, particularly scientific applications. Some 
sample application ideas include:
\begin{enumerate}
\item %1
If you have a scientific computation that runs for hours, days, or weeks, 
you could add a quick routine that outputs a snapshot of the simulation  
after every time step. This would provide an easy way to check 
the status of your long-running simulation without the overhead of 
starting up Matlab, Tecplot, etc. With X-forwarding, you could even 
check your simulation snapshot remotely over an SSH shell on free 
WiFi at Panera! :-) 
\item %2
You could write a small program that generates animation frames from  
your simulation data. Then, remotely start the program over a command 
line shell, let it run on its own, and collect the results later. Again, no 
overhead of Tecplot or Matlab, and no user interaction required! 
(You could then combine the movie frames into an AVI animation using 
EasyBMPtoAVI. See \texttt{http://easybmptoavi.sourceforge.net}.) 
\item %3
Import patient data (e.g., MRI imagery) into a 
patient-tailored simulation or for medical analysis. 
\item %4
Create arbitrary starting shapes for level set methods 
based on your own hand-drawn BMP files. This eliminates the 
artificial restriction of only being able to simulate  
shapes whose level set functions that can readily be described 
with a formula. 
\item %5
Interface your science applications with imaging equipment. 
\item %6
Use \EasyBMP{} as a quick testbed for new image processing 
ideas.
\item %7
Import snapshots from a webcam, compare them, and use \EasyBMP{} 
to remotely detect changes in a room.
\item %8
Import textures for raytracing and OpenGL programs. 
\item %9
Save screenshots of OpenGL programs. Or an X program. Or ... 
\item %10
Create nice graphics for a system monitoring utility.
\end{enumerate}

\section{Installing and Using the \EasyBMP{} Library}
Installing the \EasyBMP{} library is easy.  Simply copy all the 
\texttt{*.h} and \texttt{*.cpp} files to the directory of your project.  
Alternatively, copy all the header files anywhere in your compiler's path. You 
should have the following files: 
\begin{enumerate}
% {$\bullet\hspace{.125in}$}{}
\item \texttt{EasyBMP.h}
\item \texttt{EasyBMP\_DataStructures.h}
\item \texttt{EasyBMP\_BMP.h}
\item \texttt{EasyBMP\_VariousBMPutilities.h}
\item \texttt{EasyBMP.cpp}
\end{enumerate}
along with a code sample in the \texttt{sample} directory. \\

To use the \EasyBMP{} library, simply include the 
\texttt{EasyBMP.h} file via 
\begin{center}
\texttt{\#include "EasyBMP.h"}
\end{center}
Note that if you have copied all the \EasyBMP{} source files 
to your compiler path, you may not need the quotes, but 
rather brackets: 
\begin{center}
\texttt{\#include <EasyBMP.h>}
\end{center}
Compile your source code as you normally would along with the single 
\texttt{EasyBMP.cpp} file; you don't have to 
link to anything. For instance, to compile the code example above with 
g++, use 
\begin{center}
\texttt{g++ -o ColorBMPtoGrayscale ColorBMPtoGrayscale.cpp EasyBMP.cpp}
\end{center}

A sample project with a makefile is included in the 
\texttt{sample} directory. It demonstrates how to compile 
\EasyBMP{} with a makefile. Please see the project website at 
\texttt{http://easybmp.sourceforge.net} for further compiling 
instructions, particularly for Microsoft Visual Studio and 
makefile tutorials. 

\section{A Few Words on the BMP file format}
Any BMP file begins with a \emph{file header}, which contains 
information on the file size and data location. The 
file header is  followed by an \emph{info header}, which has information on the 
dimensions and color depth of the file. All this information is 
stored in the first 54 bytes of the file. \\

After that, the data is stored. There are two basic 
storage schemes for BMP files. If the bit depth is 8 or 
less (256 colors or fewer), the colors are stored in a 
table of (red,green,blue,alpha) values immediately after the 
info header. In this \emph{indexed} format, each pixel 
in the image refers to the position of a color in the color 
table. In essence, this storage technique is 
``painting by number.'' The benefit is that each pixel 
requires little space, but each of the 256 colors can 
attain the full range allowed on modern machines. (One of 
$256^3$ colors with $256$ values of transparency.) \\

The other scheme involves storing the red, green, and blue 
data at every pixel, along with (possibly) the alpha value. 
This format is very intuitive and allows for the full 
range of colors for all pixels. However, the resulting files 
are substantially larger than for 1, 4, and 8 bpp (bits per pixel)
files. \\

An intermediate storage scheme is to allocate two consecutive 
bytes per every pixel (16 bpp bit depth). In this scheme, 5 bits are 
allocated to red and blue, and 5 or 6 bits are allocated to green. 
The green channel is given preference over the other channels 
because the human eye is most sensitive to green, followed by 
red and then blue; allocating the extra bit to green makes the 
most rational use of the space when considering the human viewer.\\

\EasyBMP{} internally converts all files to 32 bpp for 
ease and consistency of handling. In particular, this makes it 
easy to write add-on functions that work on all bit-depth files. 
\EasyBMP{} converts back to the original bit depth when saving 
the file. \\

The coordinate system of a BMP file has its origin in the top left corner 
of the image. The $(i,j)^{\textrm{th}}$ pixel is $i$ pixels from the left 
and $j$ pixels from the top. \EasyBMP{} indexes pixels with the 
same coordinate system. See Figure \ref{BitmapCoordinateSystem}. 

% \begin{minipage}[c]{4in}
\begin{figure}[!hbtf]
% \framebox{
\begin{center}
\begin{picture}(200,155)(0,40)
\put(10,50){\line(1,0){180}}
\put(10,50){\makebox(0,-3)[t]{\texttt{(0,N-1)}}}
\put(10,185){\line(1,0){180}}
\put(10,185){\makebox(0,0)[b]{\texttt{(0,0)}}}

\put(10,50){\line(0,1){135}}
\put(190,185){\makebox(3,0)[b]{\texttt{(M-1,0)}}}
\put(190,50){\line(0,1){135}}
\put(190,50){\makebox(3,-3)[t]{\texttt{(M-1,N-1)}}}

\put(75,50){\dashbox{5}(0,135)[b]{}}
\put(75,186){\makebox(0,0)[b]{\texttt{i}}}
\put(10,150){\dashbox{5}(180,0)[b]{}}
\put(10,150){\makebox(0,0)[r]{\texttt{j}}}
\put(75,150){\circle*{6}}
\put(75,150){\makebox(0,-3)[tl]{\texttt{(i,j)}}}
\end{picture}\\
% }
\begin{minipage}{0.75\linewidth}
\caption{Coordinate system for an \texttt{M} $\times$ \texttt{N} BMP image. The 
black circle indicates the $\texttt{(i,j)}^{\textrm{th}}$ pixel
in the coordinate system.\label{BitmapCoordinateSystem}}
\end{minipage}
\end{center}
\end{figure}
% \end{minipage}


\section{Basic Bitmap Operations}
As of Version 0.55, \EasyBMP{} has a unified interface for all 
bit depths. To initialize a new \texttt{BMP} object, simply 
declare it:
\ExampleBox{\texttt{// Declare a new bitmap object}\\
\texttt{BMP AnImage;}}

When you declare a \texttt{BMP} image, you will have a $ 1 \times 1$ 
blank 24 bpp (bits per pixel) bitmap image. 
Next, set the size and bit depth of the image. You can do 
this either by reading an existing bitmap image or setting 
this information manually, as below:

\ExampleBox{\texttt{BMP AnImage;}\\
\texttt{// Set size to 640 $\times$ 480}\\
\texttt{AnImage.SetSize(640,480);}\\
\texttt{// Set its color depth to 8-bits}\\
\texttt{AnImage.SetBitDepth(8);}\\
\texttt{// Declare another BMP image}\\
\texttt{BMP AnotherImage;}\\
\texttt{// Read from a file}\\
\texttt{AnotherImage.ReadFromFile("sample.bmp");}
}

To check the bit depth, width, and height of a 
\texttt{BMP} object, use:

\ExampleBox{\texttt{BMP AnImage;}\\
\texttt{AnImage.ReadFromFile("sample.bmp");}\\
\texttt{cout << "File info:" << endl;}\\
\texttt{cout << AnImage.TellWidth() << " x " << AnImage.TellHeight()}\\
\texttt{\phantom{cout}  << " at " << AnImage.TellBitDepth() << " bpp" << endl;}
}

\EasyBMP{} also provides a simple routine to compute and display the 
number of colors:

\ExampleBox{\texttt{BMP AnImage;}\\
\texttt{AnImage.ReadFromFile("sample.bmp");}\\
\texttt{cout << "colors: " << AnImage.TellNumberOfColors() << endl;}}

Note that for a 32 bpp file, we don't regard two colors that differ 
only in the alpha channel as different colors; this function will 
state that 32 bpp and 24 bpp files have the same number of colors. \\

The bit depth and dimensions of a bitmap can be changed at any time:

\ExampleBox{\texttt{BMP AnImage;}\\
\texttt{AnImage.ReadFromFile("sample.bmp");}\\
\texttt{\slash\slash{} Change the bit-depth}\\
\texttt{AnImage.SetBitDepth(8);}\\
\texttt{AnImage.SetBitDepth(24);}\\
\texttt{\slash\slash{} Change the size}\\
\texttt{AnImage.SetSize(1024,768);}
}

Note that whenever the bit depth is changed, any existing color table is 
erased. Likewise, whenever the size is changed, all pixels are deleted. \\

To access pixels, use \texttt{RGBApixel* operator()(int,int)}: 

\ExampleBox{\texttt{BMP AnImage;}\\
\texttt{AnImage.ReadFromFile("sample.bmp");}\\
\texttt{// show the color of pixel (14,18)}\\
\texttt{cout << "(" << (int) AnImage(14,18)->Red\phantom{ab} << "," }\\
\texttt{\phantom{cout << "("} << (int) AnImage(14,18)->Green << "," }\\
\texttt{\phantom{cout << "("} << (int) AnImage(14,18)->Blue\phantom{a} << "," }\\
\texttt{\phantom{cout << "("} << (int) AnImage(14,18)->Alpha << ")" << endl; }\\
\texttt{// Change this pixel to a blue-grayish color}\\
\texttt{AnImage(14,18)->Red\phantom{ab} = 50; }\\
\texttt{AnImage(14,18)->Green = 50; }\\
\texttt{AnImage(14,18)->Blue\phantom{a} = 192; }\\
\texttt{AnImage(14,18)->Alpha = 0; }
}

You can also access pixels using \texttt{RGBApixel GetPixel(int,int)} 
and \texttt{bool SetPixel(int,int,RGBApixel)}:

\ExampleBox{\texttt{BMP AnImage;}\\
\texttt{AnImage.ReadFromFile("sample.bmp");}\\
\texttt{// show the color of pixel (14,18)}\\
\texttt{RGBApixel Temp = AnImage.GetPixel(14,18);}\\
\texttt{cout << "(" << (int) Temp.Red\phantom{ab} << "," }\\
\texttt{\phantom{cout << "("} << (int) Temp.Green << "," }\\
\texttt{\phantom{cout << "("} << (int) Temp.Blue\phantom{a} << "," }\\
\texttt{\phantom{cout << "("} << (int) Temp.Alpha << ")" << endl; }\\
\texttt{// Change this pixel to a blue-grayish color}\\
\texttt{Temp.Red = 50; Temp.Green = 50; Temp.Blue = 192; Temp.Alpha = 0;}\\
\texttt{AnImage.SetPixel(14,18,Temp); }
}

Lastly, to save to a file, use \texttt{WriteToFile(char*)}:

\ExampleBox{\texttt{BMP AnImage;}\\
\texttt{AnImage.ReadFromFile("sample.bmp");}\\
\texttt{AnImage.WriteToFile("copied.bmp");}
}

\section{Advanced Usage: Modifying the Color Table}
\label{ColorTableSection}
In \EasyBMP{}, the \texttt{BMP} class will automatically generate 
a ``Windows standard'' color table whenever one is required (for 1, 4, 
and 8 bpp files). However, it is possible to access and modify 
individual colors in the color table. If you want to see what 
the $n^{th}$ color in the color table and change another, use the following:
\ExampleBox{\texttt{BMP SomeImage;}\\
\texttt{// Set the bit depth to 8 bpp}\\
\texttt{SomeImage.SetBitDepth(8);}\\
\texttt{// Get the 40th color}\\
\texttt{RGBApixel SomeColor = SomeImage.GetColor(40);}\\
\texttt{// Display the color}\\
\texttt{cout << (int) SomeColor.Red \phantom{ab}<< ","}\\
\texttt{\phantom{cout} << (int) SomeColor.Green << ","}\\
\texttt{\phantom{cout} << (int) SomeColor.Blue \phantom{a}<< ","}\\
\texttt{\phantom{cout} << (int) SomeColor.Alpha << endl;}\\
\texttt{// Change the 14th color to red}\\
\texttt{RGBApixel NewColor;}\\
\texttt{NewColor.Red\phantom{ab} = 255;}\\
\texttt{NewColor.Green = 0;}\\
\texttt{NewColor.Blue\phantom{a} = 0;}\\
\texttt{NewColor.Alpha = 0;}\\
\texttt{SomeImage.SetColor(14,NewColor);}
}
You can reset the color table to the ``Windows standard'' color table at any time:
\ExampleBox{\texttt{BMP SomeImage;}\\
\texttt{// Set the bit depth to 8 bpp}\\
\texttt{SomeImage.SetBitDepth(8);}\\
\texttt{// Change the 14th color to red}\\
\texttt{RGBApixel NewColor;}\\
\texttt{NewColor.Red\phantom{ab} = 255;}\\
\texttt{NewColor.Green = 0;}\\
\texttt{NewColor.Blue\phantom{a} = 0;}\\
\texttt{NewColor.Alpha = 0;}\\
\texttt{SomeImage.SetColor(14,NewColor);}\\
\texttt{// Reset the color table}\\
\texttt{SomeImage.CreateStandardColorTable();}
}
To maintain previously-provided functionality, \EasyBMP{} still 
provides a grayscale color table generator:
\ExampleBox{\texttt{BMP SomeImage;}\\
\texttt{// Set the bit depth to 8 bpp}\\
\texttt{SomeImage.SetBitDepth(8);}\\
\texttt{// Create a grayscale color table}\\
\texttt{CreateGrayscaleColorTable( SomeImage );}
}
Notice that in the example, the argument is a \texttt{BMP} object. \\

If you want to modify a color table for a \texttt{BMP} file, you can 
do so by using the \texttt{SetColor(int,RGBApixel)} member function. 
To avoid messy error messages, any color table operation should do nothing 
when applied to a 16, 24, or 32 bpp file. Consider this example: 

\ExampleBox{\texttt{void CreateRedColorTable( BMP\& InputImage ) }\\
\texttt{\{}\\
\texttt{\phantom{a}int BitDepth = InputImage.TellBitDepth();}\\
\texttt{\phantom{a}if( BitDepth != 1 \&\& BitDepth != 4 \&\& BitDepth != 8 )\{ return; \}}\\
\texttt{\phantom{a}int NumberOfColors = IntPow(2,BitDepth); int i;}\\
\texttt{\phantom{a}ebmpBYTE StepSize;}\\
\texttt{\phantom{a}if( BitDepth != 1 )}\\
\texttt{\phantom{a}\{ StepSize = 255/(NumberOfColors-1); \}}\\
\texttt{\phantom{a}else}\\
\texttt{\phantom{a}\{ StepSize = 255; \}}\\
\texttt{\phantom{a}for( i=0 ; i < NumberOfColors ; i++)}\\
\texttt{\phantom{a}\{}\\
\texttt{\phantom{ab}RGBApixel Temp;} \\
\texttt{\phantom{ab}Temp.Red\phantom{ab} = i*StepSize;}\\
\texttt{\phantom{ab}Temp.Green = 0;}\\
\texttt{\phantom{ab}Temp.Blue\phantom{a} = 0;}\\
\texttt{\phantom{ab}Temp.Alpha = 0;}\\
\texttt{\phantom{ab}InputImage.SetColor(i,Temp);}\\
\texttt{\phantom{a}\}}\\
\texttt{\}}
}
To call this new function, you would do this:
\ExampleBox{\texttt{BMP RedImage;}\\
\texttt{RedImage.ReadFromFile("sample.bmp");}\\
\texttt{RedImage.SetBitDepth(8);}\\
\texttt{CreateRedColorTable( RedImage );}
}

\section{Tools for Horizontal and Vertical Resolution}
In some circumstances, it may be useful to specify the 
horizontal and vertical resolution of the machine where a 
bitmap image was generated. The \texttt{BMP} provides for 
this possibility with the \texttt{biXPelsPerMeter} and 
\texttt{biYPelsPerMeter} data members, although most files 
simply set these to 0, indicating that the system default 
should be used instead. \\

\EasyBMP{} now supports setting these numbers via the 
\texttt{SetDPI()} function:

\ExampleBox{\texttt{BMP SomeImage};\\
\texttt{SomeImage.ReadFromFile("sample.bmp");}\\
\texttt{// Set horizontal resolution to 64 DPI and vertical to 32 DPI}\\
\texttt{SomeImage.SetDPI(64,32);}}

The default vertical and horizontal resolutions are 96 dpi, 
and they are indicated by the globals
\ExampleBox{\texttt{int DefaultXPelsPerMeter = 3780;}\\
\texttt{int DefaultYPelsPerMeter = 3780;}}

You can change the default behavior by setting these global  
variables to your own preferred values. The conversion is 
pretty simple:
\begin{eqnarray*}
\texttt{DefaultXPelsPerMeter} 
& = &
\left( \texttt{DesiredDefaultDPI} \textrm{ dots / inch}\right) \cdot 
\left( \frac{1 \textrm{ inches}}{2.54 \textrm{ cm}} \right) \cdot 
\left( \frac{100 \textrm{ cm}}{1 \textrm{ m}} \right) \\
& = & 
\frac{\texttt{DesiredDefaultDPI}}{0.0254} \textrm{ dots / meter}.
\end{eqnarray*}
The \texttt{DefaultYPelsPerMeter} can be changed similarly.

\section{\EasyBMP{} and Warning Messages}
You can turn the error and warning messages on or off in 
\EasyBMP{}; this is most useful for applications where terminal 
output is not desired. 

\ExampleBox{\texttt{// turn the error messages off}\\
\texttt{SetEasyBMPwarningsOff();}\\
\\
\texttt{// The following line would ordinarily give an out-of-range warning.}\\
\texttt{SomeImage( SomeImage.TellWidth() , SomeImage.TellHeight() )->Red = 0; }\\
\\
\texttt{// turn the messages back off}\\
\texttt{SetEasyBMPwarningsOn();} \\
\\
\texttt{// query the current message status}\\
\texttt{cout << "EasyBMP warning status: " << GetEasyBMPwarningState() << endl;}
}

\section{\EasyBMP{} and Metadata}
\EasyBMP{} currently supports reading \texttt{BMP} files with 
metadata by skipping it; in particular, \EasyBMP{} does not 
preserve or save metadata. In the future, \EasyBMP{} will 
preserve metadata occurring before and after the pixel data, 
as well as allow for the writing of additional metadata.

\section{Extra Goodies: Various Bitmap Utilities}
We have provided several sample utilities to make the 
library more immediately useful.  We shall detail some of these 
goodies here. :-). \\

The first several utilities deal with getting file information 
from existing files. 

\begin{list}{$\bullet\hspace{.125in}$}{}
\item 
\underline{\texttt{void DisplayBitmapInfo( char* szFileNameIn )}:} This routine 
displays the bitmap information from an existing bitmap. All information is 
given. (width and height of image, bit depth, etc.) 

\item 
\underline{\texttt{BMFH GetBMFH( char* szFileNameIn )}:} This 
returns a \texttt{BMFH} based on the file.  See Section 
\ref{DataTypeSection} for more information on the data structure. 

\item 
\underline{\texttt{BMIH GetBMIH( char* szFileNameIn )}:} This 
returns a \texttt{BMIH} based on the file.  See Section 
\ref{DataTypeSection} for more information on the data structure. 

\item 
\underline{\texttt{int GetBitmapColorDepth( char* szFileNameIn )}:} This routine 
returns the bit depth of the file. 
\end{list}

Other provided functions are ``cut `n' paste'' functions: they copy pixels from 
one \texttt{BMP} object to another, with or without transparency. 

\begin{list}{$\bullet\hspace{.125in}$}{}
\item 
\begin{verbatim}
void PixelToPixelCopy( BMP& From, int FromX, int FromY,  
                       BMP& To, int ToX, int ToY)
\end{verbatim}

This function copies the \texttt{(FromX,FromY)} pixel of the 
\texttt{BMP} object \texttt{From} to pixel 
\texttt{(ToX,ToY)} of the \texttt{BMP} object \texttt{To}. 

\item 
\begin{verbatim}
void PixelToPixelCopyTransparent( BMP& From, int FromX, int FromY,  
                                  BMP& To, int ToX, int ToY,
                                  RGBApixel& Transparent )
\end{verbatim}

This function copies the \texttt{(FromX,FromY)} pixel of the 
\texttt{BMP} object \texttt{From} to pixel 
\texttt{(ToX,ToY)} of the \texttt{BMP} object \texttt{To}, and it 
treats the input pixel as transparent if its color is 
\texttt{Transparent}.  Here's an example: 

\ExampleBox{\texttt{BMP Image1;}\\
\texttt{BMP Image2;}\\
\texttt{Image1.ReadFromFile("Blah.bmp");}\\
\texttt{Image2.SetSize(10,10);}\\
\texttt{RGBApixel TransparentColor;}\\
\texttt{TransparentColor.Red = 255;}\\
\texttt{TransparentColor.Green = 255;}\\
\texttt{TransparentColor.Blue = 255;}\\
\texttt{PixelToPixelCopyTransparent(Image1,3,5,Image2,0,0,TransparentColor);}}

Note that the alpha channel is ignored when considering transparency.

\item 
\begin{verbatim}
void RangedPixelToPixelCopy( BMP& From, int FromL , int FromR, int FromB, int FromT, 
                             BMP& To, int ToX, int ToY )
\end{verbatim}

This function copies a range of pixels from one image to another. It copies 
the rectangle \break \texttt{[FromL , FromR]} $\times$ \texttt{[FromB , FromT]}
in image \texttt{From} to the rectangle whose top left corner is 
\texttt{(ToX , ToY)} in image \texttt{To}. When using this function, don't forget 
that the top left corner of the image is \texttt{(0,0)} in the coordinate system! 
Also, \texttt{FromB} denotes the bottom edge of the rectangle, so 
\texttt{FromB} $>$ \texttt{FromT}.  However, if the algorithm detects that you 
accidentally reversed these numbers, it will automatically swap them for you.  Lastly, 
if the rectangle you chose to copy from image \texttt{From} overlaps the boundary 
of image \texttt{To}, it will truncate the the copy selection, rather than give a 
nasty segmentation fault. :-) 

\item 
\begin{verbatim}
void RangedPixelToPixelCopyTransparent( 
     BMP& From, int FromL , int FromR, int FromB, int FromT, 
     BMP& To, int ToX, int ToY ,
     RGBApixel& Transparent )
\end{verbatim}

This function does the same thing as the previous function, but with 
support for transparency.  As in the example for the pixel-to-pixel 
copy above, you specify a transparent color of type \texttt{RGBApixel}. 

\end{list}

To maintain backward compatibility, we provide a function 
to generate a grayscale color table.

\begin{list}{$\bullet\hspace{.125in}$}{}
\item
\underline{\texttt{bool CreateGrayscaleColorTable( BMP\& InputImage ):}}
This function creates a grayscale color table for any 1, 4, or 8 bpp 
\texttt{BMP} object. It returns \texttt{true} when successful 
and \texttt{false} otherwise. An example is given in Section 
\ref{ColorTableSection}.
\end{list}

The last function rescales an image:

\begin{list}{$\bullet\hspace{.125in}$}{}
\item
\begin{verbatim}
bool Rescale( BMP& InputImage, char mode, int NewDimension );
\end{verbatim}

There are several different modes of operation available, 
which are probably best explained by example. All scaling 
is done by bilinear interpolation. Note that the 
\texttt{mode} input is not case-sensitive.

\ExampleBox{\texttt{BMP AnImage;}\\
\texttt{AnImage.ReadFromFile("sample.bmp");}\\
\texttt{// rescale to 42\% of original size }\\
\texttt{Rescale( AnImage , 'p', 42 );}\\
\\
\texttt{// rescale (preserving aspect ratio) to width 100}\\
\texttt{Rescale( AnImage , 'W', 100 );}\\
\\
\texttt{// rescale (preserving aspect ratio) to height 100}\\
\texttt{Rescale( AnImage , 'H', 100 );}\\
\\
\texttt{// rescale (preserving aspect ratio) to fit in a 100 x 100 box}\\
\texttt{Rescale( AnImage , 'f', 100 );}
}
\end{list}

\section{Hidden Helper Functions}
\EasyBMP{} has a few helper functions that are intended primarily 
for internal use, but may also be helpful for your code. 

\begin{list}{$\bullet\hspace{.125in}$}{}
\item 
\underline{\texttt{double Square( double number )}:} This routine 
gives the square of the double \texttt{number}. It is much faster 
than using the standard library \texttt{pow(double,2.0)}. 

\item 
\underline{\texttt{int IntSquare( int number )}:} This does 
the same as the above, but with faster integer operations. 

\item 
\underline{\texttt{int IntPow( int base, int exponent )}:} Instead 
of calling \texttt{pow(x,n)} where \texttt{x} and \texttt{n} are integers, 
use this function. It's much faster. Beware against using 
negative exponents, as they aren't supported. 
\end{list}

\section{Known Bugs and Quirks}
%%%%% RESUME HERE %%%%%

As of Version \Version{}, there is one known quirk: 
there is no  
\texttt{operator=} operator defined for the \texttt{BMP} class. This 
means that you can't do code operations like this:
\ExampleBox{\texttt{BMP Sample1;}\\
\texttt{Sample1.ReadFromFile("sample1.bmp");}\\
\texttt{BMP Sample2;}\\
\texttt{Sample2.ReadFromFile("sample2.bmp");}\\
\texttt{Sample1 = Sample2;}
}

AFAIK, the biggest implication of this is that you won't want to make 
functions that return a \texttt{BMP} object. Instead, if you 
want to modify a \texttt{BMP} object in a function, you should 
pass it by reference to the function. An example of this 
can be seen in the \texttt{CreateRedColorTable(BMP\&)} sample function 
given in an earlier section. \\

\texttt{operator=} functionality may be 
added to \EasyBMP{} at a later date, but only if they do not 
interfere with stability and robustness. \\

The other ``quirk'' in \EasyBMP{} is that writing 8 bpp images 
is very slow. This is a consequence of the fact that 8 bpp 
images can be operated on just like 24 bpp images, and 
so the write routine must search for the best fit color 
for each pixel when saving. This may be improved in future 
versions of \EasyBMP{}.

\section{Future Changes}
Future changes may include:
\begin{enumerate}
\item %3
an optimal color table \texttt{GenerateOptimalColorTable()} generator function;
\item %5
eliminating the use of ceil and floor functions, as well 
conversions from double to int wherever possible.
\item 
adding support for metadata before and after the pixel data
\item 
improving file write speeds for 8 bpp files. 
\end{enumerate}

\section{Obtaining Support for \EasyBMP{} or Contacting \EasyBMP{}}
If you need support, have a feature request, or have other
feedback on \EasyBMP{}, please open a new support tracker item 
at \texttt{http://easybmp.sourceforge.net}. (See the ``support''
link on that page.) If you also desire email support, you can 
email Paul Macklin at \EmailAddress{} or indicate your email 
address in your tracker item. However, please still 
open a tracker item in such a case. \\

Lastly, Paul would love to hear back from people 
who have successfully used \EasyBMP{} in their own projects. 

\appendix

\section{Classes and BMP Data Types}
\label{DataTypeSection}
Here, we detail the various classes and data types and how to interface with them.  
\subsection{Miscellany}
Some of the data types that are used in the construction of more 
complex data types are:
\begin{center}
\begin{tabular}{l | r}
Type: & Info: \\
\hline 
\texttt{ebmpBYTE} & an unsigned character of 8 bits\\
\texttt{ebmpWORD} & an unsigned short of 16 bits\\
\texttt{ebmpDWORD} & an unsigned long of 32 bits\\
\texttt{BMFH} & a specific header format for a BMP file\\
\texttt{BMIH} & provides additional information on the BMP file
\end{tabular}
\end{center}
For additional information on the \texttt{BMFH} and 
\texttt{BMIH} classes, I highly recommend that 
you visit \\

\texttt{http://www.fortunecity.com/skyscraper/windows/364/bmpffrmt.html}.

\subsection{RGBApixel}
This data structure is exactly as their its suggests: a single pixel 
of (red,green,blue,alpha) data. This data structure is used both 
for individual pixels within an image and the color table in the 
palette. Here are the details on the data structure:

\begin{center}
\begin{tabular}{l | r}
Member: &  Function: \\
\hline 
\texttt{Blue} & blue pixel info of type \texttt{BYTE}\\
\texttt{Green} & green pixel info of type \texttt{BYTE}\\
\texttt{Red} & red pixel info of type \texttt{BYTE}\\
\texttt{Alpha} & alpha pixel info of type \texttt{BYTE}
\end{tabular}
\end{center}

\subsection{BMP}
The \texttt{BMP} class consists of all the necessary pixel information 
for a Windows bitmap file, along with file I/O routines. 
\begin{list}{$\bullet\hspace{.125in}$}{}
\item 
\underline{\texttt{int BitDepth}:} This gives the number of bits per pixel, 
i.e., the color depth. This data member is private and can only be 
accessed through \texttt{TellBitDepth} and \texttt{SetBitDepth}.
\item 
\underline{\texttt{int Width}:} This gives the width of the 
bitmap in pixels. This data member is private and can only be 
accessed through \texttt{TellWidth} and \texttt{SetSize}.
\item 
\underline{\texttt{int Height}:} This gives the height of the 
bitmap in pixels. This data member is private and can only be 
accessed through \texttt{TellHeight} and \texttt{SetSize}.

\item 
\underline{\texttt{RGBApixel** Pixels}:} This is the actual 
\texttt{Width} $\times$ \texttt{Height} array of \texttt{RGBApixel}'s. 
This data member is private and can only be accessed through 
\texttt{operator()}.

\item 
\underline{\texttt{RGBApixel* Colors}:} This is the table of colors, 
stored as \texttt{RGBApixel}'s. If the \texttt{BMP} object is 
24-bits or 32-bits, then \texttt{Colors = NULL}. This data member 
is private and can only be accessed through \texttt{GetColor} 
and \texttt{SetColor}.

\item 
\underline{\texttt{int XPelsPerMeter:}} This records the horizontal resolution in pixels per meter. This 
private data member can only be accessed through 
\texttt{SetDPI()}. This value defaults to 3780, i.e., 
96 dpi.

\item 
\underline{\texttt{int YPelsPerMeter:}} This records the vertical resolution in pixels per meter. This 
private data member can only be accessed through 
\texttt{SetDPI()}. This value defaults to 3780, i.e., 
96 dpi.

\item 
\underline{\texttt{BYTE* MetaData1:}}
This is a placeholder for future metadata before the 
pixel data. This data member is private. This is currently a 
placeholder.

\item 
\underline{\texttt{int SizeOfMetaData1:}} This 
private data member gives the size of 
\texttt{MetaData1} in bytes. This is currently a 
placeholder.

\item 
\underline{\texttt{BYTE* MetaData2:}}
This is a placeholder for future metadata after the 
pixel data. This data member is private. This is currently a 
placeholder.

\item 
\underline{\texttt{int SizeOfMetaData2:}} This 
private data member gives the size of 
\texttt{MetaData2} in bytes. This is currently a 
placeholder.

\item 
\underline{\texttt{void SetDPI( int HorizontalDPI, int VerticalDPI ):}} This function sets the 
private data members \texttt{XPelsPerMeter} 
and \texttt{YPelsPerMeter} such that the image 
has horizontal and vertical resolutions 
\texttt{HorizontalDPI} and 
\texttt{VerticalDPI} dots per inch, respectively.


\item 
\underline{\texttt{int TellBitDepth( void )}:} This 
function outputs the bit depth of the \texttt{BMP} object. 

\item 
\underline{\texttt{int TellWidth( void )}:} This 
function outputs the width of the \texttt{BMP} object. 

\item 
\underline{\texttt{int TellHeight( void )}:} This 
function outputs the height of the \texttt{BMP} object. 

\item 
\underline{\texttt{int TellNumberOfColors( void )}:} This 
function outputs the number of colors of the \texttt{BMP} object. 

\item 
\underline{\texttt{BMP()}:} This constructor creates a 
1 $\times$ 1, 24 bpp \texttt{BMP} object. 

\item 
\underline{\texttt{BMP( BMP\& Input ):}} This the copy constructor, 
which is used when \texttt{BMP} objects are passed by value, 
rather than reference.

\item 
\underline{\texttt{\~{}BMP()}:} This is the destructor. You should 
never call this; it is automatically called when a \texttt{BMP} object 
goes out of scope. 

\item
\underline{\texttt{RGBApixel GetPixel( int i, int j ) const:}} This 
is used to get the contents of the \texttt{(i,j)} pixel while 
respecting \texttt{const}, using a syntax similar to the 
color table functions below. Primarily for future use. 

\item 
\underline{\texttt{bool SetPixel( int i, int j, RGBApixel NewPixel ):}} 
This is used to set the contents of the \texttt{(i,j)} pixel 
in a syntax similar to the color table functions below. Primarily for 
future use. 

\item 
\underline{\texttt{RGBApixel* operator()(int i, int j)}:} This returns a 
pointer to the (\texttt{i,j}) pixel. 
\ExampleBox{\texttt{BMP Sample;}\\
\texttt{Sample.SetSize(10,10);}\\
\texttt{Sample(3,4)->Red = 255;}\\
\texttt{Sample(3,4)->Alpha = 0;}\\
\texttt{Sample(3,4)->Blue = Sample(3,4)->Red;}
}
If \texttt{operator()} is used on a pixel that is out of range, 
it warns the user and automatically truncates the range. 

\item 
\underline{\texttt{bool SetSize( int NewWidth, int NewHeight )}:} Use 
this to change the size of the object to \texttt{NewWidth} $\times$ 
\texttt{NewHeight}. See the example above. If \texttt{NewWidth} or 
\texttt{NewHeight} is non-positive, this function gives a warning 
and does nothing. Returns true or false to indicate success or failure.

\item 
\underline{\texttt{bool SetBitDepth( int NewDepth )}:} This function 
changes the bit depth to \texttt{NewDepth} bits per pixel. It also 
automatically creates and/or resizes the color table, if necessary. 
If \texttt{NewDepth} is not one of 1, 4, 8, 24, or 32, the function 
gives an error and does nothing. Returns true or false to 
indicate success or failure.


\item 
\underline{\texttt{bool WriteToFile( char* FileName )}:} This function 
writes the current \texttt{BMP} object to the file 
\texttt{FileName}. Returns true or false to indicate success or failure.


\item 
\underline{\texttt{bool ReadFromFile( char* FileName )}:} This function 
reads the file \texttt{FileName} into the current \texttt{BMP} object.
Returns true or false to indicate success or failure.

\item 
\underline{\texttt{RGBApixel GetColor( int ColorNumber )}:} This function 
returns color number \texttt{ColorNumber} in the color table of the 
current \texttt{BMP} object. If \texttt{ColorNumber} is out of range 
or the \texttt{BMP} does not have a color table (e.g., in the case 
of a 24-bit file), this function gives a warning and returns a 
black pixel. 

\item 
\underline{\texttt{bool SetColor( int ColorNumber , RGBApixel NewColor)}:} 
This function 
sets color number \texttt{ColorNumber} in the color table of the 
current \texttt{BMP} object to \texttt{NewColor}.  
If \texttt{ColorNumber} is out of range or the \texttt{BMP} 
does not have a color table (e.g., in the case 
of a 24-bit file), this function gives a warning and does nothing.
Returns true or false to indicate success or failure.

\item 
\underline{\texttt{bool CreateStandardColorTable( void )}:} 
This function creates a ``Windows standard'' color table for the 
\texttt{BMP} object. If the bit depth is 24 or 32, the function gives a 
warning and does nothing. Returns true or false to indicate success or failure.


\end{list}
\end{document}
